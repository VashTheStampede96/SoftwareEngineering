\section{Project management}
Il \textbf{Project Management} \`e l'insieme delle azioni atte a
\textbf{pianificare}, \textbf{monitorare} e \textbf{controllare}
progetti software. \`E necessario assicurarsi che il software
venga consegnato in tempo e che soddisfi i requisiti, rispettando
il budget.

La figura del \textbf{manager}:

\begin{description}
    \item[Pianifica]: qual \`e l'obbiettivo, quali sono i tempi
    
    \item[Organizza]: che attivit\`a svolgere, quali modelli usare
    
    \item[Rende disponibili le risorse]: selezione del personale
    
    \item[Dirige]: supportare l'esecuzione delle attivit\`a previste
    
    \item[Monitora e controlla]: verificare l'esecuzione e i risultati,
                                mantenere il focus sull'obbiettivo
                                mediante eventuali azioni correttive 
\end{description}

\subsection{Planning}

La prima attivit\`a per la gestione di un progetto software riguarda
la \textbf{pianificazione}: \`e un'attivit\`a \textbf{continua} che
segue il progetto in tutta la sua vita; il \textbf{piano} principale
si compone di piani di supporto con lo scopo di gestire la qualit\`a e
la validazione del software, il suo mantenimento, lo sviluppo del
personale, ecc \dots

Il piano iniziale prevede la stesura di \textbf{milestones} (
obbiettivi "importanti" al termine dei quali viene solitamente steso
un \textit{report}), \textbf{deliverables} (risultati che possono
essere consegnati direttamente al cliente) e \textbf{scheduling}
(organizzare i vari task, i tempi, dove deve lavorare il personale,
ecc \dots).

\subsection{Scheduling}

L'attivit\`a di scheduling prevede la \textbf{suddivisione} del progetto
in task semplici, \textbf{stimarne} le risorse e il tempo necessari al
completamento, decidere come organizzare i task \textbf{concorrentemente}
per poter utilizzare in modo ottimale la \textit{workforce},
la \textbf{minimizzazione delle dipendenze} per evitare stalli causati
da ritardi nello sviluppo.

Al fine di aiutare la comprensione dello scheduling vengono utilizzati
\textbf{Bar charts} (per confrontare i tempi dei task rispetto al
calendario) e \textbf{Activity networks} (un grafo per visualizzare le
\textit{dependencies} e il \textbf{critical path}, ovvero il percorso
pi\`u lungo sulla network).\\~\\

\begin{figure}[H]
\begin{center}
    \includegraphics[width=350pt]{img/BarChart.png}
    
    \vspace{2cm}
    
    \includegraphics[width=350pt]{img/ActivityNetwork.png}

    \caption*{Bar chart e Activity network}
\end{center}
\end{figure}

\subsection{Gestione del Rischio}

Forse una delle attivit\`a pi\`u importanti, la gestione del rischio
mira a \textbf{identificare} i possibili rischi e \textbf{pianificare}
attivit\`a con lo scopo di minimizzarne la probabilit\`a di verifica
e l'effetto sul progetto. \`E indispensabile \textbf{anticipare}
e quindi \textbf{evitare} (\textit{il pi\`u possibile}) i rischi
e capirne l'impatto negativo sul progetto, il prodotto e il business.\\

I rischi si possono classificare secondo due caratteristiche:
\textbf{tipo di rischio} (tecnico, organizzativo, \dots) e \textbf{cos\`e
affetto dal rischio} (scheduling, qualit\`a, performance, \dots).\\

Il \textbf{processo di gestione del rischio} prevede:
\begin{description}
    \item[Identificazione] dei possibili rischi, attraverso un
                \textit{brainstorming} o l'esperienza del manager,
                una \textit{checklist} dei rischi pi\`u comuni 
     
    \item[Analisi] dei rischi per capire con che \textbf{probabilit\`a}
                possono verificarsi (molto bassa $\rightarrow$ molto alta)
                e quali \textbf{conseguenze} avranno sul progetto
                (insignificante $\rightarrow$ catastrofico)

    \item[Planning] di attivit\`a con lo scopo di evitare o minimizzare
                    conseguenze: ogni rischio viene gestito con una
                    strategia (\textbf{avoidance} per abbassare la
                    probabilit\`a, \textbf{minimization} per
                    diminuirne gli effetti, \textbf{contingency}
                    per affrontarlo quando si presenta)
    
    \item[Monitoring] del progetto per verificare che il planning
                    abbia funzionato, per controllare come si
                    evolvono i rischi e per, eventualmente, definire
                    nuovi planning 
\end{description}

Al fine di identificare i primi rischi vengono solitamente utilizzati
degli \textbf{indicatori} che riguardano la \textbf{tecnologia}
(se c'\`e mancanza di supporto software o vengono rilevati numerosi
problemi tecnici), il \textbf{personale} (se non c'\`e poca relazione
tra i membri del team), gli \textbf{strumenti} (se c'\`e riluttanza
all'utilizzo di CASE), la \textbf{stima dei tempi} (se si verificano
ritardi).

\begin{figure}[H]
    \centering
    \includegraphics[width=350pt]{img/RiskManagement.png}
    \caption*{Il processo di gestione del rischio}
\end{figure}