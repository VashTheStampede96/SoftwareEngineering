\chapter{System Engineering}

Un \textbf{Sistema} \`e una collezione di componenti interconnessi
che lavorano insieme per il raggiungimento di un obbiettivo comune.\\
Nel nostro contesto, un sistema include software e hardware e viene
utilizzato dalle persone. Le componenti di un sistema dipendono dalle
altre e le loro propriet\`a e il loro comportamento sono intrinsecamente
interlegati.\\

I problemi affrontati dal \textbf{System Engineering} riguardano
lo sviluppo di sistemi complessi designati per durare nel tempo
in un \textbf{ambiente in continua evoluzione}. Il software sta
avendo sempre pi\`u importanza nella costruzione di sistemi complessi
e spesso viene visto come un problema in quanto (a volte) ritarda
lo sviluppo di grossi progetti.

Per questo bisogna avere una visione pi\`u generale del sistema
prima di gettarsi nella produzione del software.\\

Le \textbf{propriet\`a complessive} di un sistema sono quelle che
emergono da una visione pi\`u ampia del sistema stesso e dipendono
\textbf{non solo} dalle propriet\`a delle singole componenti ma anche
dalle propriet\`a che derivano dalle loro interazioni.\\~\\

\textit{Esempi}
\begin{description}
    \item[Peso complessivo] dipende solo dal peso
            dei singoli componenti
    \item[Affidabilit\`a] dipende \textbf{anche} dalle interazioni
    \item[Usabilit\`a] \`e una propriet\`a complessa che dipende
            anche da \textbf{chi} usa il software~e \textbf{dove}
\end{description}
\newpage

Le propriet\`a complessive si dividono in:
\begin{description}
    \item[Functional] riguardano la relazione \textbf{input/output}
            del sistema (ex. Se la bicicletta \`e costruita come si deve
            allora \`e un mezzo di trasporto valido)
    \item[Non-functional] riguardano il \textbf{comportamento} del sistema
            nel suo ambiente di utilizzo; affidabilit\`a, performance,
            sicurezza (\textit{safety}) e protezione (\textit{security});
            sono propriet\`a critiche in quanto il \textit{non}
            raggiungimento di determinati livelli rende il sistema
            inutilizzabile
\end{description}

% \begin{figure}[H]
%     \centering
%     \includegraphics[width=\linewidth]{img/SEProcess.png}
% \end{figure}

\section{Evoluzione di un sistema}
I sistemi complessi hanno una vita lunga e devono evolversi costantemente
seguendo i nuovi requisiti. Questa evoluzione \`e \textbf{costosa}:
deve essere \textit{analizzata} dal punto di vista tecnico e del business
e possono nascere problemi inaspettati a causa dell'\textit{interazione} tra
vecchi e nuovi componenti, la \textit{struttura} del sistema pu\`o
corrompersi con l'evoluzione dello stesso.

I sistemi che \textbf{devono} essere mantenuti vengono definiti \textbf{legacy}.

\section{System procurement}

\begin{figure}[H]
        \centering
        \includegraphics[width=\linewidth]{img/SysProcurement.png}
        \caption*{Schema per l'acquisizione di un sistema}
\end{figure}

Per \textit{system procurement} si intende l'\textbf{acquisizione} di un
sistema per un'organizzazione con lo scopo di soddisfarne le necessit\`a.
Solitamente \`e necessario stabilire a priori le \textbf{specifiche}
e il design dell'architettura per:
\begin{itemize}
        \item Scrivere un contratto che stabilisca \textbf{cosa}
                andr\`a fatto e \textbf{come}
        \item Capire se esistono sistemi gi\`a sviluppati
                (\textit{off the shelf}) per ci\`o
                che serve (di solito meno costoso rispetto a costruire
                un sistema da zero)
\end{itemize}

In generale, che sia \textit{off the shelf} o che sia creato da zero,
lo sviluppo/mantenimento del software viene delegato a terzi.
Il contraente si occuper\`a di delegare a sua volte parte del lavoro
a sub-contraenti.

\begin{figure}[h]
        \centering
        \includegraphics[width=300pt]{img/ContractorModel.png}
        \caption*{Modello contraente/subcontraente}
\end{figure}

\section{Concetti chiave}

\begin{itemize}
        \item Concetto generale di \textbf{Sistema} come insieme di
                componenti interlegati che cooperano per un obbiettivo
                in comune
        \item \textbf{Problematiche} dell'Ingegneria dei Sistemi:
                sviluppare sistemi complessi per durare nel tempo
        \item \textbf{Emergent properties}, \textbf{Shall-not properties}
        \item \textbf{Legacy}
        \item Processo e fasi principali del SysEng
        \item \textbf{System procurement} e modello
        \item \textbf{Contractor e subcontractor} (contraente)
\end{itemize}