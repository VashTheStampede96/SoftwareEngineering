\newpage
\section{Le attivit\`a fondamentali nel SW development}

Nello sviluppo di un prodotto software sono ricorrenti quattro
attivit\`a, presenti in ogni modello di processo.

\begin{figure}[H]
    \centering
    \includegraphics[width=350pt]{img/VModel.png}
    \caption*{Il \textbf{V-Model} che integra le 4 attivit\`a
    fondamentali.}
\end{figure}

\subsection{Analisi dei requisiti e specifiche del software}

Ricordiamo che l'analisi dei \textbf{requisiti} vuole determinare
le \textbf{esigenze del cliente} mentre la definizione delle
\textbf{specifiche} stabilisce quali \textbf{servizi deve offrire
il software}.

\begin{figure}[H]
    \centering
    \includegraphics[width=300pt]{img/RequirementEngProcess.png}
    \caption*{Il processo di ingegnerizzazione del software.}
\end{figure}


\subsection{Progettazione e sviluppo}

Dopo un'accurata analisi dei requisiti e delle specifiche \`e
necessario \textbf{progettare} il software e successivamente
\textbf{svilupparlo} in un programma eseguibile. L'attivit\`a
di progettazione viene svolta \textit{topdown}:

\begin{description}
    \item[Architettura]: come suddividere il sistema principale
                        in sotto-sistemi e moduli

    \item[Interfaccia]: come comunicano i vari moduli e sotto-sistemi
                        tra di loro

    \item[Componenti]:  Come devono essere sviluppati internamente
                        i sotto-sistemi

    \item[Strutture Dati]: Quali strutture dati utilizzare e come
                            rappresentarle in un database

    \item[Algoritmo]
\end{description}

La progettazione viene documentata attraverso un insieme di modelli
grafici, i pi\`u utilizzati sono: \textbf{UML}, \textbf{Reti di Petri},
\textbf{MBD}, \dots

Per quanto riguarda lo sviluppo, si compone di due attivit\`a:
\textbf{programmazione} (dove viene scritto il software)
e \textbf{debugging} (dove vengono rilevati e risolti i \textit{faults}).

\begin{figure}[H]
    \centering
    \includegraphics[width=350pt]{img/DebuggingProcess.png}
    \caption*{Il processo di debugging}.
\end{figure}

\subsection{Verifica e Validazione}
La \textbf{verifica} consiste nella dimostrazione che il software
\`e conforme alle specifiche (\textit{fa quello per cui \`e stato
pensato}). La \textbf{validazione} consiste nell'accertarsi che
il software soddisfi i requisiti (\textit{fa quello di cui il cliente
ha bisogno}).

Il processo di verifica e validazione lavora \textit{bottomup},
dalle singole unit\`a integrando via via moduli e sotto-sistemi,
fino a testare il sistema completo.

\begin{figure}[H]
    \centering
    \includegraphics[width=350pt]{img/VerValProcess.png}
    \caption*{Il processo di testing}
\end{figure}

\ \\
\subsection{Evoluzione}
I \textbf{requisiti} possono cambiare (per strategie di mercato,
per la concorrenza, perch\'e il cliente ha deciso cos\`i) e
il software deve quindi essere modificato; diventa quindi fondamentale
gestire efficacemente ogni parte del prodotto: dalla progettazione,
allo sviluppo e infine alla manutenzione. Come supporto vengono
spesso utilizzati software \textbf{CASE} (\textit{computer-aided
software engineering}) che offrono la possibilit\`a di gestire
con facilit\`a alcuni aspetti del ciclo di vita. Da ricordare per\`o
che il software engineering \`e un'attivit\`a che richiede
\textbf{creativit\`a} e \textit{teamwork} (non facilmente automatizzabili).

\subsection{Concetti chiave}

TODO