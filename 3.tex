\chapter{Processi Software}

Un processo \`e un insieme strutturato di attivit\`a (\textit{routine})
atte a trasformare \textit{qualcosa} da una forma ad un'altra.
Per quanto riguarda i \textbf{processi software} possiamo
caratterizzarli in:
\begin{description}
    \item[Primari] specificare, progettare, produrre, mantenere ed
                estendere il prodotto SW (\textit{ricorda le fasi
                per la creazione di un Software})
    \item[Supporto] eseguiti parallelamente per garantire la qualit\`a
                e il successo del progetto
    \item[Gestione] necessarie per gestire il progetto di sviluppo
                del prodotto SW
\end{description}

\section{Processi Primari}

Il \textbf{Ciclo di Vita} \`e uno schema che specifica come sono
definiti e organizzati i vari \textit{task} coinvolti
nella progettazione, sviluppo e manutenzione di un prodotto.
Decide \textbf{COSA} fare e \textbf{QUANDO}.

Lo schema \`e \textbf{astratto} e \textbf{generale} (ricopre diverse
casistiche senza entrare in contesti specifici e senza considerare
i dettagli implementativi).

Viene organizzato in \textbf{fasi} che definiscono i passi principali
e \textbf{task} che definiscono i compiti da svolgere in ogni fase.\\

La \textbf{Metodologia di Sviluppo} definisce \textbf{COME} svolgere
i vari task in modo \textbf{effettivo} e specifica in modo
\textbf{concreto} e \textbf{preciso} i dettagli implementativi considerando
anche il contesto in cui si trova la metodologia.

\begin{figure}[H]
    \centering
    \includegraphics[width=300pt]{img/PrimaryProcess.png}
    \caption*{Rappresentazione grafica di un processo software}
\end{figure}

\subsection{Concetti chiave}

\begin{itemize}
    \item Cos'\`e un \textbf{processo software} e come caratterizzarlo
    \item Processi primari: \textbf{Lifecycle}, \textbf{Methodology}
\end{itemize}

\section{Modello di un processo}

Richiamando la definizione dei capitoli precedenti, il
\textbf{modello di un processo software} \`e una rappresentazione semplificata
di un processo software presentato da un particolare punto di vista.

In particolare, i modelli ci permettono di parlare di \textbf{quali}
attivi\`a devono essere svolte e in che ordine, tralasciando i dettagli
implementativi. Permettono di descrivere:
\begin{description}
    \item[Prodotti] risultanti dalle attivit\`a del processo
    \item[Ruoli] e responsabilit\`a delle persone coinvolte
    \item[Pre/Post conditions] delle attivit\`a
\end{description}

Attraverso un modello, il processo pu\`o essere \textbf{plan-driven}
(le attivit\`a seguono quello che \`e stato pianificato a priori)
oppure \textbf{agile} (dove il \textit{planning} \`e incrementale
e pu\`o essere modificato durante lo sviluppo per riflettere gli
interessi del \textit{customer}). Nella pratica vengono per\`o utilizzati
modelli \textbf{ibridi}.\\

Ci preoccuperemo ora di descrivere alcuni \textit{process models}.
\newpage

\subsection{Waterfall}

\begin{figure}[H]
    \centering
    \includegraphics[width=300pt]{img/Waterfall.png}
    \caption*{Rappresentazione grafica del modello a cascata}
\end{figure}

Il \textbf{Modello a Cascata} prevede il susseguirsi ordinato delle
varie fasi di sviluppo:
\begin{description}
    \item[Analisi dei Requisiti]: cercare di capire \textbf{cosa}
            serve al customer
    \item[Design del sistema e del software]
    \item[Implementazione e testing]: viene scritto il codice e
            testato singolarmente
    \item[Integrazione e testing]: vengono integrate le varie
            componenti del progetto e ne vengono testate le
            interazioni
    \item[Mantenimento]: fase di evoluzione del software e
            supporto del customer    
\end{description}

\textbf{Idealmente} ad ogni fase si passa alla successiva,
\textbf{nella pratica} si ritorna spesso indietro per successive
modifiche al processo.

\begin{description}
    \item[Vantaggi:]\ 
    \begin{itemize}
        \item Idealmente non si torna indietro
        \item Il passaggio alla fase successiva prevede la compilazione
                di un documento che permette al committente di
                mantenersi aggiornato sullo sviluppo del progetto
                e di assicurarsi che il lavoro venga svolto secondo
                i suoi interessi
        \item Il passaggio da una fase all'altra \`e scandito da
                documenti (piace ai manager) e si conosce sempre
                la fase in cui ci si trova (a favore della
                \textbf{visibilit\`a} e della \textbf{gestibilit\`a})
    \end{itemize} 
    \item[Svantaggi:]\ 
    \begin{itemize}
        \item Poco flessibile ai cambiamenti sui requisiti
        \item Nella pratica si ritorna alle fasi precedenti e questo
                \`e \textbf{tanto pi\`u costoso quanto pi\`u indietro
                si ritorna}
        \item Usato principalmente per grossi progetti di SysEng
    \end{itemize} 
\end{description}

\subsection{Prototyping}

\begin{figure}[H]
    \centering
    \includegraphics[width=300pt]{img/EvolutionaryModel.png}
    \caption*{Rappresentazione del Modello Prototyping (vengono
                prodotte diverse versioni del sistema via via
                sempre pi\`u precise)}
\end{figure}

Il \textbf{Prototyping Model} o \textbf{Evolutionary Development}
cerca di dare una soluzione al problema dei cambi frequenti
sui requisiti. L'idea \`e quella di produrre diverse versioni
successive del sistema per avere pi\`u informazioni possibili
sul risultato finale che si vuole ottenere e su \textbf{come} ottenerlo.

\begin{description}
    \item[Exploratory]: l'obbiettivo \`e quello di partire dai
                requisiti pi\`u chiari per poi evolvere il sistema
                per comprendere i restanti requisiti lavorando a
                contatto con il cliente
    \item[Thrown-away]: partire dai requisiti meno chiari per
                sviluppare un sistema \textbf{prototipo} per comprendere
                meglio i requisiti; al termine viene \textit{gettato}
                il prototipo 
\end{description}

\begin{description}
    \item[Vantaggi:]\ 
        \begin{itemize}
            \item Riduzione del costo per le modifiche sui requisiti
            \item Pi\`u partecipazione del cliente su cosa \`e stato
                    implementato
        \end{itemize}
    \item[Svantaggi:]\ 
        \begin{itemize}
            \item Poca \textbf{visibilit\`a} (\textit{a che punto siamo?})
            \item Poca \textbf{gestibilit\`a} (\textit{quanto manca?})
            \item Poca \textbf{strutturazione}
        \end{itemize}
\end{description} 

Viene utilizzato, in genere, per sistemi interattivi \textbf{medio-grandi},
\textbf{tecnologie innovative}, \textbf{parti di grossi sistemi},
\textbf{short lifetime systems}.

\ \\~\\~\\

\subsection{Formal systems development}
Il modello si basa sulla trasformazione di una specificazione
\textbf{matematica formale} in un programma eseguibile. Queste
trasformazioni preservano la correttezza ed \`e quindi semplice
dimostrare che il programma \`e conforme alle specifiche.
Ottimo modello \textbf{teorico} ma nella pratica \textbf{inutilizzabile}.\\

\ \\
\begin{figure}[H]
    \centering
    \includegraphics[width=300pt]{img/FormalSystem1.png}
    \\~\\~\\
    \includegraphics[width=300pt]{img/FormalSystem2.png}
    \caption*{Fasi e trasformazioni del modello Formal System.}
\end{figure}

\ \\~\\
Richiede \textbf{skills specializzate} ed \textbf{esperienza} per
poter applicare la tecnica, inoltre \`e difficile specificare
alcuni aspetti del sistema, come la \textbf{user interface}.
Viene generalmente applicato a \textbf{sistemi critici} dove la
\textit{safety} e la \textit{security} vengono sviluppate ancora prima
che il sistema venga messo ad operare.

\newpage
\subsection{Reuse-oriented development}
Generalmente utilizzato quando il sistema si costruisce integrando
componenti gi\`a esistenti o quando viene adottato un sistema
\textit{off-the-shelf}. Il modello ha preso sempre pi\`u piede
con l'avvento della \textbf{OOP} e dell'\textbf{opensource}.

\ \\
\begin{figure}[H]
    \centering
    \includegraphics[width=350pt]{img/ReuseOrientedModel.png}
    \caption*{Le fasi del Reuse-oriented model.}
\end{figure}

Al classico modello generico viene aggiunta una fasa di \textit {
system design} attraverso i componenti gi\`a esistenti che andranno
poi testati e integrati.

\ \\~\\
\subsection{Incremental Development}
Qualunque sia il modello o il progetto, i requisiti possono evolvere
(e quasi certamente lo faranno). \`E necessario dunque gestire
piccoli e mirati \textbf{incrementi} alle funzionalit\`a del prodotto.
I requisiti vengono \textbf{prioritarizzati}. Quando un incremento
viene iniziato i requisiti vengono congelati, tuttavia per i successivi
incrementi possono ancora evolvere.

\ \\
\begin{figure}[H]
    \centering
    \includegraphics[width=350pt]{img/IncrementalDevelopment.png}
    \caption*{Una volta definiti i requisiti vengono assegnati
        agli \textit{increments}, sviluppati, integrati e validati;
        se il sistema non \`e ancora pronto si ritorna allo sviluppo.}
\end{figure}

\newpage
\begin{description}
    \item[Vantaggi:]\ 
    \begin{itemize}
        \item Il cliente pu\`o utilizzare il sistema anche se non
                \`e completo (sfruttando i requisiti a priorit\`a
                alta gi\`a sviluppati)
        \item Gli \textit{early increments} possono essere sfruttati
                come \textit{prototype}
        \item \textbf{minor rischio di fallimento del progetto}
                (\textit{perch\`e viene sviluppato incremento dopo
                    incremento})
        \item I requisiti a priorit\`a alta vengono sviluppati nei
                primi incrementi e verranno quindi testati pi\`u volte 
    \end{itemize} 
    \item[Svantaggi:]\ 
    \begin{itemize}
        \item Senza \textit{tutti} i requisiti \`e difficile decidere
                come organizzare gli incrementi (\textit{quanto
                granulati devono essere?})
        \item Pu\`o richiedere moduli contrattuali non-standard oppure
                contratti successivi
    \end{itemize} 
\end{description}

\ \\
\subsection{Iterative Development}
Una variante dello sviluppo incrementale prevede la costruzione del
\textbf{sistema completo} e solo successivamente l'implementazione
degli incrementi ad ogni nuova \textit{release}.

\ \\
\begin{figure}[H]
    \centering
    \includegraphics[width=320pt]{img/DifferenceIncrementalIterative.png}
    \caption*{Le differenze tra \textit{Incremental} e \textit{Iterative}
            development: il primo prevede l'inserimento di nuove funzionalit\`a,
            il secondo la modifica delle precedenti.}
\end{figure}

\subsection{Extreme Programming}
Un nuovo approccio basato sullo sviluppo e la consegna di
\textbf{piccoli incrementi} delle funzionalit\`a. \`E caratterizzato da
\textbf{miglioramenti costanti del codice}, il \textbf{coinvolgimento
dell'utente nello sviluppo} e \textit{pairwise programming}.\footnote{
Due sviluppatori sulla stessa macchina: uno scrive e l'altro controlla
che non ci siano errori.}

\subsection{Spiral Development}
\`E pi\`u un approccio generale che un modello di \textit{lifecycle}.
Il processo \`e rappresentato come una spirale invece che una
sequenza con backtracking: ogni ciclo della spirale rappresenta una
fase del processo e \textbf{non esistono fasi fissate}, viene invece
scelto in base a ci\`o che serve. I \textbf{rischi} sono cos\`i risolti
durante il processo (partendo dalle attivit\`a che hanno il rischio
\textbf{pi\`u alto}).\\~\\

Il modello a spirale si compone di 4 fasi:
\begin{description}
    \item[Identificazione dell'obbiettivo]
    \item[Valutazione e riduzione del rischio]
    \item[Sviluppo e validazione]
    \item[Pianificazione] della successiva fase della spirale
\end{description}

\begin{figure}[H]
    \centering
    \includegraphics[width=350pt]{img/SpiralModel.png}
\end{figure}

\section{Il rischio}
Il \textbf{rischio} denota la precisa probabilit\`a di una specifica
eventualit\`a, convenzionalmente \textbf{negativa}, che pu\`o
influenzare alcune caratteristiche del sistema che possono emergere
in futuro.\\~\\

\subsection{Le cause del fallimento di un progetto di sviluppo SW}
\begin{itemize}
    \item Requisiti incompleti o vaghi
    \item Mancanza di coinvolgimento dell'utente
    \item Mancanza di risorse
    \item \textit{Skills} non adatte al lavoro
    \item Aspettativa irrealistica
    \item \textbf{Management}, \textit{planning} o \textit{cost estimation}
            inadeguati
    \item Mancanza di comunicazione
    \item Architettura inadeguata
    \item Cambiamenti dei requisiti o delle specifiche
\end{itemize}

\subsection{Concetti chiave}

TODO

\newpage
\section{Le attivit\`a fondamentali nel SW development}

Nello sviluppo di un prodotto software sono ricorrenti quattro
attivit\`a, presenti in ogni modello di processo.

\begin{figure}[H]
    \centering
    \includegraphics[width=350pt]{img/VModel.png}
    \caption*{Il \textbf{V-Model} che integra le 4 attivit\`a
    fondamentali.}
\end{figure}

\subsection{Analisi dei requisiti e specifiche del software}

Ricordiamo che l'analisi dei \textbf{requisiti} vuole determinare
le \textbf{esigenze del cliente} mentre la definizione delle
\textbf{specifiche} stabilisce quali \textbf{servizi deve offrire
il software}.

\begin{figure}[H]
    \centering
    \includegraphics[width=300pt]{img/RequirementEngProcess.png}
    \caption*{Il processo di ingegnerizzazione del software.}
\end{figure}


\subsection{Progettazione e sviluppo}

Dopo un'accurata analisi dei requisiti e delle specifiche \`e
necessario \textbf{progettare} il software e successivamente
\textbf{svilupparlo} in un programma eseguibile. L'attivit\`a
di progettazione viene svolta \textit{topdown}:

\begin{description}
    \item[Architettura]: come suddividere il sistema principale
                        in sotto-sistemi e moduli

    \item[Interfaccia]: come comunicano i vari moduli e sotto-sistemi
                        tra di loro

    \item[Componenti]:  Come devono essere sviluppati internamente
                        i sotto-sistemi

    \item[Strutture Dati]: Quali strutture dati utilizzare e come
                            rappresentarle in un database

    \item[Algoritmo]
\end{description}

La progettazione viene documentata attraverso un insieme di modelli
grafici, i pi\`u utilizzati sono: \textbf{UML}, \textbf{Reti di Petri},
\textbf{MBD}, \dots

Per quanto riguarda lo sviluppo, si compone di due attivit\`a:
\textbf{programmazione} (dove viene scritto il software)
e \textbf{debugging} (dove vengono rilevati e risolti i \textit{faults}).

\begin{figure}[H]
    \centering
    \includegraphics[width=350pt]{img/DebuggingProcess.png}
    \caption*{Il processo di debugging}.
\end{figure}

\subsection{Verifica e Validazione}
La \textbf{verifica} consiste nella dimostrazione che il software
\`e conforme alle specifiche (\textit{fa quello per cui \`e stato
pensato}). La \textbf{validazione} consiste nell'accertarsi che
il software soddisfi i requisiti (\textit{fa quello di cui il cliente
ha bisogno}).

Il processo di verifica e validazione lavora \textit{bottomup},
dalle singole unit\`a integrando via via moduli e sotto-sistemi,
fino a testare il sistema completo.

\begin{figure}[H]
    \centering
    \includegraphics[width=350pt]{img/VerValProcess.png}
    \caption*{Il processo di testing}
\end{figure}

\ \\
\subsection{Evoluzione}
I \textbf{requisiti} possono cambiare (per strategie di mercato,
per la concorrenza, perch\'e il cliente ha deciso cos\`i) e
il software deve quindi essere modificato; diventa quindi fondamentale
gestire efficacemente ogni parte del prodotto: dalla progettazione,
allo sviluppo e infine alla manutenzione. Come supporto vengono
spesso utilizzati software \textbf{CASE} (\textit{computer-aided
software engineering}) che offrono la possibilit\`a di gestire
con facilit\`a alcuni aspetti del ciclo di vita. Da ricordare per\`o
che il software engineering \`e un'attivit\`a che richiede
\textbf{creativit\`a} e \textit{teamwork} (non facilmente automatizzabili).

\subsection{Concetti chiave}

TODO
