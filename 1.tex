\chapter{Introduzione}
Per \textbf{Software} si intende un \textit{computer program}
con associata \textbf{documentazione}.
\begin{itemize}
    \item sviluppato per un \textit{customer} in particolare
          o per il \textit{general market}
    \item \textbf{generic} oppure \textbf{customized}
    \item creato da zero (\textit{from scratch}) oppure da software gi\`a
          scritto
\end{itemize}

L'\textbf{Ingegneria del Software} \`e una disciplina \textbf{ingegneristica}
che si occupa di tutti gli aspetti legati alla \textbf{produzione} e
all'\textbf{utilizzo} del software. \\
Essendo una disciplina ingegneristica deve sempre tenere in considerazione
le limitazioni economiche e organizzative del real world.
\\

Per \textbf{Specificazione} si intende \textbf{cosa} deve (dovrebbe)
fare il software: nel \textit{generic} lo decide il \textit{developer},
nel \textit{customized} decide il cliente.
\\

\textbf{CS vs SE}: teoria, formalismo, \textit{fundamentals} vs
software utile, sviluppo pratico, \textit{real world problems}.

\textbf{SysE}: si occupa di tutti gli aspetti dei sistemi, hardware,
software e processi. SE \`e una parte del SysEng.
\\~\\

\noindent
\framebox{\parbox{\dimexpr\linewidth-2\fboxsep-2\fboxrule}{\itshape%
\textbf{IMPORTANZA DEL SOFTWARE ENGINEERING}: individui e
societ\`a dipendono sempre di pi\`u da sistemi software avanzati,
\`e quindi necessario produrre \textbf{velocemente}
software \textbf{affidabile}}}

\newpage
\textbf{Processo Software}: insieme di attivit\`a il cui obbiettivo
\`e lo sviluppo e l'evoluzione del software:
\begin{itemize}
      \item \textbf{Specificazione} customer e SEng decidono le
            caratteristiche del software da produrre
      \item \textbf{Sviluppo} viene prodotto il sistema software
            (design, progettazione e realizzazione)
      \item \textbf{Validazione} controllare che il software
            soddisfi i requisiti
      \item \textbf{Evoluzione} modifica del software per riflettere
            i cambiamenti del cliente e i requisiti di mercato
\end{itemize}

\textbf{Modello di un processo software}: rappresentazione semplificata
di un processo software presentato da un particolare punto di vista
(\textit{i.e.} workflow, dataflow, role/action).
Modelli spesso utilizzati sono \textit{Waterfall, Evolutionary development,
Formal transformation, Integration from reusable components}.\\

Ad oggi, le principali problematiche affrontate dal SE riguardano:
\begin{enumerate}
      \item Sitemi \textit{legacy} che \`e \textbf{necessario}
            mantenere ed aggiornare
      \item Eterogeneit\`a dei sistemi distribuiti
      \item Consegna, sicurezza e fiducia
\end{enumerate}

\newpage
\section{Concetti chiave}

\begin{itemize}
      \item Importanza del SE: paesi sviluppati dipendono dal SW
      \item Distribuzione dei costi: 60\% sviluppo (analisi,
            progettazione, programmazione), 40\% testing
      \item Importanza della documentazione
      \item Ragionamento analitico e sintetico
      \item Processo SW, Modello di un SW-process
      \item Parametri di qualit\`a del software: manutentibilit\`a,
            sicurezza, protezione, efficienza
      \item Sfide del SE: sistemi legacy, eterogeneit\`a HW/SW,
            sviluppo rapido e consistente
\end{itemize}